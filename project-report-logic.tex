\documentclass{article}
\usepackage{amsmath}
\usepackage{venturis2}
\usepackage{mathtools}
\begin{document}
\begin{center}
\begin{large}
{\bf Taking the SVD of a Matrix}
\end{large}
\end{center}

\vskip0.3in
\begin{center}
{\large  Oscar Alvarez}\\
Texas Woman's University\\
June 2016
\end{center}
\vskip0.6in
\section{Solving for the SVD}

\subsection{The SVD of a matrix.}

The SVD is defined as the singular value decomposition of a matrix. 

We are going to be dealing with eigenvalues and eigenvectors so let us say that we do not have a perfectly square matrix. (We have an nxm instead of an nxn matrix).
https://preview.overleaf.com/public/swznrzxzdmjf/images/a052cb2b9fb18e98742752a010a4ae4a6e4db884.jpeg

When we break down a matrix into the SVD, the matrix is divided into three different individual matrices. The equation would look like this:
\begin{equation}
\label{SVD}
A = U \Sigma V^T
\end{equation}

\subsection{Solving for the SVD of a matrix.}
\begin{itemize}

\item
The first step would be to find either\quad $A^TA$\quad  or\quad $AA^T$. We have the option to use either one in the case that we do not have a perfectly square matrix. If we began with a 2x3 matrix, we would rather solve for $AA^T$ because the matrix that turns out will be a 2x2 - instead of a 3x3.
For example if we had 
\begin{equation*}
C=\begin{bmatrix*}[r]
2 & -2 & 1 \\ 5 & 1 & 4
\end{bmatrix*}
\end{equation*}\quad then, 
\begin{equation*}
CC^T=\begin{bmatrix*}[r]
2 & 2 & 1 \\ 5 & 1 & 4
\end{bmatrix*} \begin{bmatrix*}[r]
2 & 5 \\ -2 & 1 \\ 1 & 4
\end{bmatrix*} = \begin{bmatrix*}[r]
9 & 12 \\ 12 & 24
\end{bmatrix*}.
\end{equation*}
This 2x2 matrix will be a lot easier to work with when solving for its eigenvalues and eigenvectors.
\end{itemize}


\subsubsection{Example using a 2x2 matrix}

We will now solve for the SVD of $A_{2x2}$.
\\
Let us say

\begin{equation*}
A=\begin{bmatrix*}[r]
2 & 2 \\ -1 & 1
\end{bmatrix*}
\end{equation*}

We begin by solving for $A^TA$ or $AA^T$

\begin{equation}
A^TA=\begin{bmatrix*}[r]
2 & -1  \\ 2 & 1
\end{bmatrix*} \begin{bmatrix*}[r]
2 & 2  \\ -1 & 1
\end{bmatrix*} = \begin{bmatrix*}[r]
5 & 3 \\ 3 & 5
\end{bmatrix*} = Q 
\end{equation}

We will now need to find the eigenvalues and vectors of $Q$. We will use this equation to solve for the eigenvalues first. Take note that this equation only works when solving for 2x2 matrices.
\begin{equation}
\label{Eigensolver}
\lambda ^2-\text{trace($Q$)$ \lambda $ + det($Q$)}=0 
\end{equation}

We will begin by finding the \quad trace($Q$). \quad The trace($Q$) is equal to the sum of the diagonals of $Q$.
\begin{equation*}
Q=\begin{bmatrix*}[r]
5 & 3 \\ 3 & 5
\end{bmatrix*}\quad\quad
\text{then,}\quad\quad
\text{trace($Q$)}=5+5=10
\end{equation*}


Next, we will solve for the det($Q$)
\begin{equation*}
\text{det($Q$)}=\left(5 \cdot 5\right)-\left(3 \cdot 3\right) =16
\end{equation*}

Using equation \ref{Eigensolver}, we now have
\begin{equation*}
\lambda ^2-10 \lambda +16=0.
\end{equation*}


Solving the equation will give us our eigenvalues.
\begin{equation*}
( \lambda -2)( \lambda - 8)=0.
\end{equation*}
\begin{equation}
\label{eigenvalues}
\lambda_{1}=2 \hspace{2cm} \lambda_{2}=8.
\end{equation}

Once we have our eigenvalues, we need to solve for our eigenvectors.
\\
Using
$$
(Q-I_{2}\lambda)=
\begin{bmatrix}
5-\lambda & 3 \\ 3 & 5-\lambda
\end{bmatrix},
$$
with
$$
\lambda_{1}=2 \hspace{2cm} \text{and} \hspace{2cm} \lambda_{2}=8
$$

we get

$$
v_{1}=\begin{bmatrix*}[r]
-1 \\ 1
\end{bmatrix*} \hspace{1.5cm} \text{and} \hspace{1.5cm} v_{2}=\begin{bmatrix*}
1 \\ 1
\end{bmatrix*}
$$

We will now normalize the eigenvectors.
\begin{equation*}
\parallel v_{1} \parallel_{2} = \frac{1}{\sqrt{2}}\begin{bmatrix*}[r]
-1 \\ 1
\end{bmatrix*}\hspace{1.5cm} \text{and} \hspace{1.5cm} \parallel v_{2} \parallel_{2} = \frac{1}{2\sqrt{2}}\begin{bmatrix*}
1 \\ 1
\end{bmatrix*}
\end{equation*}


Create a matrix with the eigenvectors. This matrix will represent our $V$ matrix which will be transposed in equation \ref{SVD}.
\begin{equation}
\label{Vmatrix}
V=\begin{bmatrix*}[r]
-\frac{1}{\sqrt{2}} & \frac{1}{2\sqrt{2}} \\
\frac{1}{\sqrt{2}} & \frac{1}{2\sqrt{2}}
\end{bmatrix*}
\end{equation}

We will need a different equation when solving for the $U$ matrix. To do this we will have to use our eigenvalues that we have already solved for (\ref{eigenvalues}).

This is the equation that we will need to solve for $U$.
\begin{equation}
u_{i}=\frac{1}{\sigma}Av_{i}
\end{equation}
\\
To find $\sigma$, we take the square root of our respective eigenvalues. 
$$
\sigma_{1}=\sqrt{2} \hspace{1.5cm} \text{and} \hspace{1.5cm} \sigma_{2}=\sqrt{8}=2\sqrt{2}
$$
\\
Now we plug in our respective values.\
\\
$$
u_{1}=\frac{1}{\sqrt{2}}
\begin{bmatrix*}[r]
2 & 2 \\
-1 & 1
\end{bmatrix*}
\begin{bmatrix*}[r]
-\frac{1}{\sqrt{2}} \\
\frac{1}{\sqrt{2}}
\end{bmatrix*}=\begin{bmatrix*}[r]
0 \\
1
\end{bmatrix*}
$$
and
$$
u_{2}=\frac{1}{2\sqrt{2}}
\begin{bmatrix*}[r]
2 & 2 \\
-1 & 1
\end{bmatrix*}
\begin{bmatrix*}[r]
\frac{1}{2\sqrt{2}} \\
\frac{1}{2\sqrt{2}}
\end{bmatrix*}=\begin{bmatrix*}[r]
\frac{1}{2} \\
0
\end{bmatrix*} 
$$
Our completed $U$ matrix:
\begin{equation}
\label{Umatrix}
U=\begin{bmatrix*}[r]
0 & \frac{1}{2} \\
1 & 0
\end{bmatrix*}
\end{equation}
\\
To create our $ \Sigma $ matrix, we simply place our respective $ \sigma_{i} $s on the diagonals
\begin{equation*}
\Sigma =\begin{bmatrix*}[l]
\sigma_{1} & 0 \\
0 & \sigma_{2}
\end{bmatrix*}
\end{equation*}
so that
\begin{equation}
\label{Sigmamatrix}
\Sigma =\begin{bmatrix*}[r]
\sqrt{2} & 0 \\
0 & 2\sqrt{2}
\end{bmatrix*}.
\end{equation} \\
If you look back at (\ref{Umatrix}) (\ref{Sigmamatrix}) and (\ref{Vmatrix}), we now have all of our matrices that make up the SVD of $A$, our original matrix. If we look back at equation (\ref{SVD}) we can now construct our equation.
\begin{equation}
A=\begin{bmatrix*}[r]
0 & \frac{1}{2} \\
1 & 0
\end{bmatrix*}
\begin{bmatrix*}[r]
\sqrt{2} & 0 \\
0 & 2\sqrt{2}
\end{bmatrix*}
\begin{bmatrix*}[r]
-\frac{1}{\sqrt{2}} & \frac{1}{2\sqrt{2}} \\
\frac{1}{\sqrt{2}} & \frac{1}{2\sqrt{2}}
\end{bmatrix*}^T =\begin{bmatrix*}[r]
2 & 2 \\ -1 & 1
\end{bmatrix*}
\end{equation}
We now have the SVD of a $2x2$ matrix.


\end{document}
