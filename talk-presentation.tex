\documentclass{beamer}

\usetheme{Copenhagen}


%\usepackage{tikz}
%\usetikzlibrary{decorations.pathmorphing}
\usepackage[english]{babel}

\usepackage{amsmath}
\usepackage{verbatim}
\usepackage{graphicx}
\usepackage[final]{mcode}
\usepackage{listings}
\usepackage{booktabs}
\definecolor{maroon}{rgb}{0.522, .035, .157}
\setbeamercolor{structure}{fg=maroon!90!black}
\usepackage{amscd}


\title{Digital Image Manipulation:\\ Modern Applications of Linear Algebra}
\author{Oscar Alvarez and Hannah Grant}
\institute{Department of Mathematics and Computer Science \\
Texas Woman's University\\
Advisor: Dr. Alicia Machuca}
\date{April 18, 2017}

\begin{document}

\begin{frame}
\titlepage
\end{frame}

\begin{frame}{Overview}
\tableofcontents
\end{frame}

\section{Project 1}
\subsection{Image Blending Using Convex Combinations}
\begin{frame}{Using Linear Algebra ``in real life"}
\begin{itemize}
\item
It is important to introduce students to real life applications of linear algebra
\vspace{.20cm}
\begin{itemize}
\item
Motivates learning
\vspace{.20cm}
\item
Develops interest
\end{itemize}
\vspace{.20cm}
\item
Google, Netflix, NFL, data mining, etc.\\
\begin{itemize}
\item
Websites that you may like
\item
New movie suggestions from ones you have already seen 
\end{itemize}
\end{itemize}
\begin{columns}
\begin{column}{.5\textwidth}
\begin{itemize}
\item
MATLAB\\
\begin{itemize}
\item
Software used\\
\end{itemize}
\end{itemize}
\end{column}
\begin{column}{.5\textwidth}
\centering
\includegraphics[scale=.17]{matlab.PNG}
\end{column}
\end{columns}
\end{frame}


\begin{frame}{MATLAB}
\begin{columns}
\begin{column}{.5\textwidth}
\includegraphics[width=\textwidth]{matlab.PNG}
\end{column}
\begin{column}{.5\textwidth}
\begin{itemize}
\item
Mathematics Engine Software
\item
Used by scientists and engineers around the world
\item
Uses a matrix-based programming language for mathematical computation
\end{itemize}
\end{column}
\end{columns}
\end{frame}

\begin{frame}{Image as a matrix using MATLAB}
\vspace{.5cm}
\begin{columns}
\begin{column}{.5\textwidth}
\centering
\includegraphics[scale=.16]{sinmatriximage.jpg}
\end{column}
\begin{column}{.1\textwidth}
$$
\huge
=
$$
\end{column}
\begin{column}{.5\textwidth}
\tiny
$$
\begin{bmatrix}
0 & 0 & 1 & 0 & 1 & 0 & 1 & 0 & 1 & 0 \\
1 & 0 & 1 & 0 & 1 & 0 & 1 & 0 & 1 & 0 \\
1 & 0 & 0 & 0 & 1 & 0 & 1 & 0 & 1 & 0 \\
1 & 0 & 1 & 0 & 1 & 0 & 1 & 0 & 1 & 0 \\
1 & 0 & 1 & 0 & 0 & 0 & 1 & 0 & 1 & 0 \\
1 & 0 & 1 & 0 & 1 & 0 & 1 & 0 & 1 & 0 \\
1 & 0 & 1 & 0 & 1 & 0 & 0 & 0 & 1 & 0 \\
1 & 0 & 1 & 0 & 1 & 0 & 1 & 0 & 1 & 0 \\
1 & 0 & 1 & 0 & 1 & 0 & 1 & 0 & 0 & 0 \\
1 & 0 & 1 & 0 & 1 & 0 & 1 & 0 & 1 & 0 \\
\end{bmatrix}
$$
\end{column}
\end{columns}
\vspace{.5cm}
\normalsize
Where 0 is black and 1 is white.\\
For a gray scale image, the pixel value is between 0 and 1.

\end{frame}
\begin{frame}{Convex Combinations}
\begin{itemize}
\item
Convex combinations
\large
\begin{equation*}
I =\alpha J+(1- \alpha)B 
\end{equation*}\\
\normalsize
\begin{itemize}
\item
$J$ and $B$ are matrices
\item
$0\leq\alpha\leq1$
\item
$\alpha+(1-\alpha)=1$
\end{itemize}
\item
Applying convex combinations to image blending, where J and B are images.
\end{itemize}
\end{frame}

\begin{frame}{Convex Combinations and Image Blending}
\begin{columns}
\begin{column}{.35\textwidth}
\includegraphics[scale=.150]{original.jpg}\\
\tiny \centering Image J
\end{column}
\begin{column}{.35\textwidth}
\includegraphics[scale=.03]{3.JPG}\\
\tiny \centering Image B
\end{column}
\end{columns}
\vspace{.2cm}
\large When $\alpha$=.40\\
\vspace{.1cm}
\centering
\normalsize
\begin{columns}
\begin{column}{.35\textwidth}
\includegraphics[scale=.153]{squirreltimesalpha.jpg}\par
\centering(a)\,\,$\alpha J$
\end{column}
\begin{column}{.35\textwidth}
\includegraphics[scale=.153]{operatimesalphaminus.jpg}\par
\centering(b)\,\,$(1-\alpha)B$
\end{column}
\begin{column}{.35\textwidth}
\includegraphics[scale=.153]{badblend.jpg}\par
\centering\,(c)\,\,$\alpha J+(1-\alpha)B$
\end{column}
\end{columns}
\end{frame}


\begin{frame}{Convex Combinations and Image Blending}
\centering\includegraphics[scale=.045]{2.png}
\centering\includegraphics[scale=.045]{3.JPG}\\
\centering\includegraphics[scale=.045]{Blended_3.png}
\end{frame}

\begin{frame}{MATLAB Program}
\centering
\includegraphics[scale=.5]{Capture.JPG}
% \small
% \begin{lstlisting}
% clear

% sqrl = imread('squirrel.jpeg');
% BWsqrl = rgb2gray(sqrl);
% J = double(BWsqrl);

% bgbooks = imread('operahouse.jpeg');
% BWbgbooks = rgb2gray(bgbooks);
% B = double(BWbgbooks);

% for alpha=[.40] 
%   Blended = alpha*J(:,:)+(1-alpha)*B(:,:);
%   figure;
%   imshow(uint8(Blended))
% end 
% \end{lstlisting}
\end{frame}

\section{Project 2}
\subsection{Image Compression Using the SVD (Singular Value Decomposition)}
\begin{frame}{The SVD (Singular Value Decomposition)}
$$A=U \Sigma V^T$$
\vspace{.6cm}
\tiny
\begin{equation*}
A = \begin{bmatrix}
u_{11} & u_{12} & . & . & . & u_{1k} \\
u_{21} & . & & & & \\
. & & . & & & \\
. & & & . & & \\
. & & & & & \\
u_{i1} & & & & & u_{ik}
\end{bmatrix}
\begin{bmatrix}
s_{11} & 0 & . & . & . & 0 \\
0 & s_{22} & & & & \\
. & & . & & & \\
. & & & . & & \\
. & & & & & \\
0 & & & & & 0
\end{bmatrix}
\begin{bmatrix}
v_{11} & v_{12} & . & . & . & v_{1k} \\
v_{21} & . & & & & \\
. & & . & & & \\
. & & & . & & \\
. & & & & & \\
v_{i1} & & & & & v_{ik}
\end{bmatrix}^{T}
\end{equation*}
\normalsize
\begin{itemize}
\item
The matrix $U$ is formed by the eigenvectors of $AA^T$.
\item
$ \Sigma $ is the matrix with the singular values of $A$ on the diagonal.
\item
Matrix $V$ is formed by the normalized eigenvectors of $A^TA$.
\end{itemize}
\end{frame}

\begin{frame}{The SVD (Singular Value Decomposition)}
\begin{itemize}
\item
How can we apply the SVD?
\vspace{.35cm}
\begin{itemize}
\item
Singular Values
\begin{itemize}
\item
Values that define the distance in relationship between related matrices.
\end{itemize}
\vspace{.35cm}
\item
Rank
\begin{itemize}
\item
The number of linearly independent rows a matrix contains.
\end{itemize}
\end{itemize}
\end{itemize}
\vspace{.6cm}
By reducing the rank of the matrix that contains the singular values ($\Sigma$), we can eliminate values that are not as crucial.
\end{frame}

\begin{frame}{Image Compression using the SVD}
Break up an image matrix using the SVD\\
\centering
\includegraphics[scale=5]{bwtwu.png}\\
\flushleft
\includegraphics[scale=.2]{mtwu.PNG} \, =\\
\includegraphics[scale=.2]{utwu.PNG}\\
\includegraphics[scale=.2]{stwu.PNG}\\
\includegraphics[scale=.2]{vtwu.PNG}
\end{frame}

\begin{frame}{Image Compression using the SVD}
$ \Sigma $ (S) matrix contains all of the singular values.\\
\centering
\includegraphics[scale=.2]{stwu.PNG}\\
\flushleft
Reduce $ \Sigma $ matrix size retaining only the first-most, largest values.\\
\centering
\vspace{.2cm}
\includegraphics[scale=.4]{ctwu.PNG}\\
\flushleft
Reduce the size of other two matrices to multiply.\\
\begin{columns}
\begin{column}{.5\textwidth}
\centering
\includegraphics[scale=.3]{atwu.PNG}\\
\end{column}
\begin{column}{.5\textwidth}
\centering
\includegraphics[scale=.3]{etwu.PNG}\\
\end{column}
\end{columns}
\end{frame}

\begin{frame}{Image Compression using the SVD}
Multiply new matrices together to form new matrix with 3 singular values.\\

\begin{columns}
\begin{column}{.1\textwidth}
\centering
\includegraphics[scale=.3]{atwu.PNG} 
\end{column}
\begin{column}{.1\textwidth}
$\times$
\end{column}
\begin{column}{.1\textwidth}
\includegraphics[scale=.3]{ctwu.PNG}
\end{column}
\begin{column}{.1\textwidth}
\centering
$\times$
\end{column}
\begin{column}{.1\textwidth}
$\includegraphics[scale=.2]{etwu.PNG}^T$
\end{column}
\end{columns}
\flushleft
This creates our new, compressed, image.\\
\centering
\vspace{.2cm}
=\includegraphics[scale=.2]{dtwu.PNG}\\
\vspace{.2cm}
\includegraphics[scale=5]{twucomp.png}
\end{frame}

\begin{frame}{Image Compression using the SVD}
\centering
\includegraphics[scale=5]{bwtwu.png} \hspace{1cm} $\rightarrow$ \hspace{1cm}  \includegraphics[scale=5]{twucomp.png}\\
\vspace{.5cm}
Reducing the Rank of the $ \Sigma $ matrix reduces image quality but saves memory.\\
\vspace{.5cm}
This could prove useful when using high quality images on small phones with lower screen resolutions.\\
\vspace{.5cm}
The less memory your wallpaper takes up, the faster your phone can be, all-the-while retaining and displaying a good quality image.
\end{frame}

\begin{frame}{MATLAB Program}
\centering
\includegraphics[scale=.40]{SVDcode.PNG}
% \begin{lstlisting}
% clear

% inImage = imread('compressionimage.jpg');
% inImage = rgb2gray(inImage);
% inImageSVD = single(inImage);

% [U,S,V]=svd(inImageSVD);

% for k=[5 30 200] %the number singular values go here
    
%     for i = 1:k;
        
%         A(:,i) = U(:,i);
%         C(i,i) = S(i,i);
%         E(:,i) = V(:,i);
        
%     end
    
%     D = A*C*E';

%     figure;
%     imshow(uint8(D))
% end
% \end{lstlisting}
\end{frame}

\begin{frame}{Image Compression using the SVD}
\begin{columns}
\begin{column}{.2\textwidth}
\centering SVD Rank\\ 421
\includegraphics[width=\textwidth]{421.PNG}
\end{column}
\begin{column}{.2\textwidth}
\centering SVD Rank\\ 150
\includegraphics[width=\textwidth]{SVD150.PNG}
\end{column}
\begin{column}{.2\textwidth}
\centering SVD Rank\\ 30
\includegraphics[width=\textwidth]{SVD30.PNG}
\end{column}
\begin{column}{.2\textwidth}
\centering SVD Rank \\ 15
\includegraphics[width=\textwidth]{SVD15.PNG}
\end{column}
\begin{column}{.2\textwidth}
\centering SVD Rank \\5
\includegraphics[width=\textwidth]{SVD5.PNG}
\end{column}
\end{columns}
\vspace{.6cm}
*\,421 being the matrix rank of the original image
\end{frame}

\section {Tutorials}
\begin{frame}{Tutorials}
\begin{itemize}
\item
Structure
\vspace{.20cm}
\begin{itemize}
\item
Brief review of matrix manipulation and terms 
\vspace{.20cm}
\item
Introduction to mathematical concepts within the projects
\begin{itemize}
\item
Convex Combinations
\item
The SVD
\end{itemize}
\vspace{.20cm}
\item
Introduction to MATLAB exercises
\vspace{.20cm}
\item
Step by step process of the projects
\begin{itemize}
\item
Image blending
\item
Image compression
\end{itemize}
\vspace{.20cm}
\end{itemize}
\item
Student feedback
\begin{itemize}
\item
Google Forms
\end{itemize}
\end{itemize}
\end{frame}

\begin{frame}{Recap}
\vspace{.5cm}
\underline{Recap:}
\vspace{.5cm}
\normalsize
\begin{itemize}
\item
Image Blending using convex combinations
\vspace{.5cm}
\item
Image Compression using the SVD (Singular Value Decomposition)
\vspace{.5cm}
\item
Tutorials
\end{itemize}
\vspace{.5cm}
\end{frame}

\begin{frame}{Acknowledgements}
\begin{itemize}
\vspace{.3cm}
\item
Dr. Alicia Machuca, Dr. Jian Zhang, Dr. Marie-Anne Demuynck
\vspace{.5cm}
\item
TWU's Quality Enhancement Plan's Experiential Student Scholar Program
\vspace{.5cm}
\item
Lamar University and TUMC
\vspace{.5cm}
\end{itemize}
\end{frame}


\bibliographystyle{plain}
\begin{thebibliography}{10}
\begin{frame}{Bibliography}
\small{


\bibitem{}
Austin, David \textit{We Recommend a Singular Value Decomposition
} \\http://www.ams.org/samplings/feature-column/fcarc-svd.\\ Retrieved: June 2016.

\bibitem[book]{}
Axler, Sheldon \textit{Linear Algebra Done Right} (2010).
}

\bibitem[book]{}
Chartier, Tim \textit{When Life is Linear: From Computer Graphics to Bracketology} (2015).

\bibitem[book]{}
Downing, Douglas \textit{Dictionary of Mathematics Terms} (1995).

\bibitem{}
MathWorks \textit{MATLAB: Major Update} \\http://www.mathworks.com/products/matlab/.\\
Retrieved: July 2016


\end{frame}
\end{thebibliography}





\end{document}



